\chapter{Project goal and supplied code}
\label{ch:general_remarks}

During the Evolution of Speech course taught at the VUB in 2021-2022 we, Computer Science students, were introduced to this multidisciplinary field by reviewing multiple important papers of the field.
As the course was taught by Bart de Boer, who has an Oxford  University Press published book on the origins of vowel systems and many papers in the field, we also reviewed some of de Boer's papers (\cite{deBoerBook, deBoer2000, deBoer2010, deBoer2018}).
As a Computer Science student with limited linguistic knowledge, papers from de Boer using Agent-Based Modelling (ABM) techniques for studying phenomena in the field were found the most interesting (\cite{deBoer2000, deBoer2010}).
Because of this, we opted to extend upon \citet{deBoer2000}.
The exact project goal and an overview of the supplied code are given in further detail here.

%------------------------------------

\section{Project goal}
\label{sec:general_remarks_why}
As the ABM related papers were found most interesting, it was chosen to re-implement and extend the paper on self-organization in vowel systems by \citet{deBoer2000} for this project.
Whilst the original C++ source code of \citet{deBoer2000} was provided to us, his students, it was dated and not so well documented.
This was to be expected as the code was not originally meant for distribution.
To further ground our understanding of the paper and make extensions on this work easier, we have chosen to do a re-implementation in Python.
The written code is well documented, easily extendable and most importantly, publicly available under the GPL V3 license \citep{gplv3, github_project}.
This enables readers to not only easily reproduce the results of \citet{deBoer2000} and the extensions provided here but also gives them a great basis for future projects.
The latter was something we felt was lacking and feel is an important contribution of this work.
We also addressed some of the \textit{ad hoc} decisions in the original version.
To be more precise, this report provides an alternative way of converting to the bark scale and determining the effective second formant of a produced sound.
This was found to not influence the results, as is further discussed in \ref{sec:reimplementing_better}.

As computer science students, we know how important the used network structure is to models and ABMs in specific.
In the original version of the imitation game proposed by \citet{deBoer2000}, agent pairs are picked at random.
Whilst this is an understandable simplification for his work, it made us wonder if the findings hold for more complex structures.
Initially, it was considered to use scale-free networks, as we thought this would better represent a human network.
However, the actual realism of scale-free networks is debated and one of our colleagues wanted to go this route already \citep{scalefree}.
Because of this, we opted to model a small community consisting of agents with different roles and influences.
These agents also die and get replaced by new agents.
It is thus more dynamic and varying than the setting used by \citet{deBoer2000} and will test the original hypothesis of his paper further.
This hypothesis is: "The structure of vowel systems is determined by self-organization in a population under constraints of perception and production." - \citet{deBoer2000}.

%------------------------------------
\section{Important files}
\label{sec:general_remarks_files}

Accompanied by this report is a copy of the GitHub repository created for this project \citep{github_project}.
It includes all files needed to reproduce the experiments, including saved versions of the games used for figures and statistics in this report.
An overview of the most important files is given below:
\begin{itemize}
    \item \texttt{README.md}
    \begin{itemize}
        \item General information of the GitHub repository with hyperlinks to important files and documentation. 
    \end{itemize}
    \item \texttt{code-output}
    \begin{itemize}
        \item All figures generated by the provided code, some of which are used in this report.
    \end{itemize}
    \item \texttt{code/notebooks}
    \begin{itemize}
        \item Jupyter notebooks and plain py files going over different aspects of the code for this project.
        \item \texttt{1\_implementing\_de\_boer\_2000.ipynb}: step by step re-implementation of code by \citet{deBoer2000}.
        \item \texttt{imitationGameClasses.py}: all classes needed to play imitation games as specified by \citet{deBoer2000}.
        \item \texttt{2\_recreating\_de\_boer\_2000.ipynb}: step by step re-collection of results by \citet{deBoer2000}.
        \item \texttt{3\_alternative\_bark\_experiments.ipynb}: step by step re-collection of results by \citet{deBoer2000} using a less ad hoc variant of the bark converter and effective second formant weighting function.
        \item TODO TODO TODO TODO TODO TODO TODO TODO
    \end{itemize}
    \item \texttt{code/html-exports} and \texttt{documentation/installation}
    \begin{itemize}
        \item HTML export of the above discussed Jupyter notebooks, ideal for those who want to view the notebooks without installing the Anaconda environment.
        \item Install instructions for the used Anaconda environment of this project (macOS and Ubuntu).
    \end{itemize}
\end{itemize}